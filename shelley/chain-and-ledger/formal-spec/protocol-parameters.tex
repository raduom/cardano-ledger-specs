\section{Protocol Parameters}
\label{sec:protocol-parameters}

For Plutus integration, we have added the following types
(see Figure~\ref{fig:defs:protocol-parameters}):

\begin{itemize}
\item $\Language$: This represents the language name/tag (including the Plutus
version number)
\item $\ExUnits$: A term of this type has two integer values, $(mem, steps)$. The first
represents some notion of the amount of memory needed for script validation.
The second, the number of reduction steps (for some definition of a step, specific
to the script interpreter)
\item $\CostMod$: A term of this type represents the vector of coefficients used to convert \\
\texttt{resource primitives}~$\to \ExUnits$. This type is abstract and the
conversion is done by a script interpreter (thus, is opaque to the ledger rules)
\item $\Prices$: A term of type is made up of three integer values,
$\var{(pr_{init}, pr_{mem}, pr_{steps})}$. The first ($pr_{init}$) represents the initial
cost of running a script. The $pr_{mem}$ is a price for a unit of memory
used by script execution, and $pr_{steps}$ is the price per
reduction step. It is used to calculate the cost of using an amount of execution
units.
\end{itemize}

We also added several protocol parameters and accessor functions necessary for
Plutus integration.

\textbf{Plutus vs. Native Script Languages.}
All the newly added protocol parameter values have to do with limiting and calculating
the script execution cost.
These are only relevant for processing scripts that are opaque
to the ledger. That is, scripts not processed explicitly according ledger rules,
but rather by using an interpreter. Currently, this refers specifically
to Plutus scripts. Because the ledger knows only the validation result, and
nothing about their execution, it relies on the interpreter to compute resource
used during the validation process. Hence the need for these parameters,
the use of which is discussed below.

Native scripts (currently this includes only the multisignature scripts), on the
other hand, are processed entirely by the ledger rules.
Because of this, their execution cost can easily be assessed before processing them.
As a result, their running cost can be accounted for in calculating the minimum transaction
fee, and its payment verified by comparing the minimum fee to the fee the transaction
is actually paying. The $\Prices, \ExUnits$ or $\CostMod$ types are not used
in processing native scripts.

\textbf{Plutus Versions.} Each
version of Plutus (identified by a $\Language$ value) is considered a different
language. That is, the interpreter
specific to that version is invoked when a script is run. Each language
has a (version-specific) set of arguments passed to the interpreter
for script evaluation. When Goguen is released, this type should contain
terms that are tags for the multisignature script language, and $\var{plcV1Tag}$,
for the first version of Plutus being released.

The ledger is expected to be capable of running Plutus scripts of
all existing versions at all times. This means maintaining any ledger
data nedeed by any version. Introducing a new version of Plutus
into the set of supported versions is a hard fork, as the actual ledger rules
and code must be updated manually to use the new interpreter.

\textbf{Plutus Script Evaluation Determinism.}
A change in data passed to the interpreter for
validating a given script can cause a change in the validation result. Such
a change may occur between the time a transaction is submitted by the wallet
and the time the block containing it is processed. Data passed to the interpreter
includes the validator, redeemer, information about the transaction carrying
the script, and some data currently part of the ledger or protocol parameters.
The exact set of arguments may be different for different versions.
It is necessary to maintain a predictable
(deterministic) validation outcome over this period (between transaction
submission and script processing).

To guarantee a deterministic outcome,
any data passed to the interpreter must be
the same as it was at the time the transaction carrying this script was
constructed.
Because of this requirement, the carrying transation includes the hash of any such data.
This hash is compared with
a hash of the same subset of ledger or parameter data computed at the time the transaction is being
processed. If these hashes do not match, the interpreter is never invoked.
This check is part of the two-phase validation model for saving time and computation
resources during transaction processing (see Section~\ref{sec:utxo}).

Currently, the only additional data passed to the interpreter (and therefore must
be hashed during transaction processing) is in the protocol parameters. The
function $\fun{hashLanguagePP}$ selects the protocol parameters relevant to
the a given set of languages and computes their hash.
See Figure~\ref{fig:defs:protocol-parameters}.

The only parameter passed to the interpreter (at this time, for any version) is the cost model
corresponding to that version, so it is also the only parameter included
in the hash needed for this determinism-preserving comparison. For the rule
where this comparison is performed, see Section~\ref{sec:block-body-trans}.

\textbf{Plutus Script Evaluation Cost Model and Prices.}
As one of the script validation parameters, each language also has a
cost model. A cost model is a set of the values used to convert resource
primitives used during script validation into the
more abstract $\ExUnits$. As this conversion is done by the Plutus interpreter,
so we leave the cost model type abstract in this specification.
The cost models for each version are stored in the protocol
parameters in the variable $\var{costmdls}$.

Having distinct cost models for each version will allow us to discourage users from
paying into scripts made using old versions of Plutus, which could have bugs
in them that are fixed in newer versions. Changing the conversion coefficients
of resource primitives into $\ExUnits$ can make paying into old versions
more expensive.

Note here that the calculation of the actual cost, in Ada, of running
running a script that takes $\var{exunits} \in \ExUnits$ resources to run,
is done by a formula specified in the ledger rules, and uses the parameter
$\var{prices}$. This is a parameter that applies to all Plutus
scripts and cannot be toggled for individual versions. This parameter is
convenient to change when real-world prices of things like electicity or hardware
fluctuate.

\textbf{Limiting Script Execution Costs}
The protocol parameters $\var{maxTxExUnits}$ and $\var{maxBlockExUnits}$ are
used to limit the total per-transaction and per-script resource use of all
types of scripts. Recall here that these only have non-trivial values for
(non-native) Plutus scripts.

\begin{figure*}[htb]
  \emph{Abstract types}
  %
  \begin{equation*}
    \begin{array}{r@{~\in~}lr}
      \var{plv} & \Language
      & \text{Script Language}
      \\
      \var{costm} & \CostMod & \text{coefficients for cost model} \\
    \end{array}
  \end{equation*}
  %
  \emph{Derived types}
  \begin{equation*}
    \begin{array}{r@{~\in~}l@{\qquad=\qquad}lr}
      \var{(pr_{init}, pr_{mem}, pr_{steps})}
      & \Prices
      & \Coin \times \Coin \times \Coin
      & \text {coefficients for ExUnits prices}
      \\
      \var{(mem, steps)}
      & \ExUnits
      & \N \times \N
      & \text{abstract execution units} \\
    \end{array}
  \end{equation*}
  %
  \emph{Protocol Parameters}
  %
  \begin{equation*}
      \begin{array}{r@{~\in~}lr}
        \var{costmdls} \mapsto (\Language \mapsto \CostMod) & \PParams & \text{script exec cost model}\\
        \var{prices} \mapsto \Prices & \PParams & \text{coefficients for ExUnits prices} \\
        \var{maxTxExUnits} \mapsto \ExUnits & \PParams & \text{max total tx script execution resources}\\
        \var{maxBlockExUnits} \mapsto \ExUnits & \PParams & \text{max total block script execution resources}\\
      \end{array}
  \end{equation*}
  %
  \emph{Accessor Functions}
  %
  \begin{center}
  \fun{costmdls},~\fun{maxTxExUnits},~\fun{maxBlockExUnits},~\fun{prices}
  \end{center}
  %
  \emph{Helper Functions}
  %
  \begin{align*}
    & \fun{hashLanguagePP} \in \PParams \to \powerset{(\Language)} \to \PPHash^?   \\
    & \fun{hashLanguagePP}~\var{pp}~\var{lgs} = \begin{cases}
         \fun{hash}~(\var{lgs} \restrictdom \fun{costmdls}~{pp})
                           & \text{if~} \var{lgs} \restrictdom \fun{costmdls}~{pp} \neq \{\} \\
              \Nothing & \text{otherwise} \\
      \end{cases} \\
    & \text{this calculation hashes the current protocol parameters relevant to
    the given set of versions}
  \end{align*}
  %
  \caption{Definitions Used in Protocol Parameters}
  \label{fig:defs:protocol-parameters}
\end{figure*}


\clearpage
